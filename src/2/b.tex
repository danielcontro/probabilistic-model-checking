\subsection{}

An IMDP satisfies a PCTL property if for every induced MDP, the property holds.
The induced MDP is obtained by fixing the probability distributions for
each action. Thus the induced MDP is a regular MDP, and the PCTL property can
be checked using the model checking algorithm for MDPs.

We will define $\pmin$ and $\pmax$ for IMDPs as follows:
\begin{itemize}
	\item $\pmin^{I}(s, \psi) = \inf_{\sigma \in Adv,Steps^\star \in Steps}Prob^{\sigma}_{Steps*}(s, \psi)$
	\item $\pmax^{I}(s, \psi) = \sup_{\sigma \in Adv,Steps^\star \in Steps}Prob^{\sigma}_{Steps*}(s, \psi)$
\end{itemize}
where $Steps^\star$ is a valuation of the transition probability functions of
the IMDP and $Prob^{\sigma}_{Steps*}$ is the probability of satisfying the
property $\psi$ with the transition probability functions $Steps^\star$.

Moreover we will define the satisfaction relation for PCTL properties for IMDPs:
\[
	s \vDash_I \prob_{\sim p}[\phi] \Leftrightarrow \pmin^{I}(s, \phi) \sim p \text{ if } \sim \in \{ >, \ge \}
\]
\[
	s \vDash_I \prob_{\sim p}[\phi] \Leftrightarrow \pmax^{I}(s, \phi) \sim p \text{ if } \sim \in \{ <, \le \}
\]

\subsubsection{}

\[
	Sat(\prob_{>0.4}[\Next(a \vee b)]) = \{s \in S \mid \pmin^I(s, \Next(a \vee b)) > 0.4 \}
\]
and
\[
	\pmin^I(s, \Next(a \vee b)) = \min \left\{ \sum_{s' \in Sat(a \vee b)} \mu(s') \mid (\alpha, \mathcal{M}) \in Steps(s), \mu \in \mathcal{M} \right\}
\]
\[
	Sat(a \vee b) = \{ s_2, s_3 \} \Rightarrow \underline{a \vee b} = \begin{bmatrix} 0; & 0; & 1; & 1 \end{bmatrix}
\]
Since the probability distributions intervals involve two transition per action
we will introduce a variable for each action denoting the probability of
performing one transition and with the complement the probability of performing
the other.
Namely:
\[
	Steps^\star =
	\begin{bmatrix}
		0            & 0           & \mu_0^\alpha  & 1-\mu_0^\alpha \\
		0            & \mu_0^\beta & 1-\mu_0^\beta & 0              \\
		\hline
		\mu_1^\alpha & 0           & 0             & 1-\mu_1^\alpha \\
		\hline
		0            & 0           & 1             & 0              \\
		\hline
		0            & 0           & 0             & 1              \\
	\end{bmatrix}
\]
with $\mu_0^\alpha \in [0.7;0.9]$, $\mu_0^\beta \in [0.3;0.6]$ and
$\mu_1^\alpha \in [0.3; 0.7]$.
hence we can compute the minimum probability of satisfying the property:
\begin{align*}
	\underline{\pmin^I}(\Next (a \vee b)) & = Steps^\star \cdot \underline{a \vee b}                             \\
	                                      & = \begin{bmatrix}
		                                          0            & 0           & \mu_0^\alpha  & 1-\mu_0^\alpha \\
		                                          0            & \mu_0^\beta & 1-\mu_0^\beta & 0              \\
		                                          \hline
		                                          \mu_1^\alpha & 0           & 0             & 1-\mu_1^\alpha \\
		                                          \hline
		                                          0            & 0           & 1             & 0              \\
		                                          \hline
		                                          0            & 0           & 0             & 1              \\
	                                          \end{bmatrix} \cdot \begin{bmatrix} 0 \\ 0 \\ 1 \\ 1 \end{bmatrix}
	                                      & = \begin{bmatrix}
		                                          1                \\
		                                          1 - \mu_0^\beta  \\
		                                          \hline
		                                          1 - \mu_1^\alpha \\
		                                          \hline
		                                          1                \\
		                                          \hline
		                                          1
	                                          \end{bmatrix}
\end{align*}
Extracting the minimum probability we get:
\begin{align*}
	\underline{\pmin^I}(\Next (a \vee b)) & = \begin{bmatrix}
		                                          1 - \mu_0^\beta  \\
		                                          1 - \mu_1^\alpha \\
		                                          1                \\
		                                          1
	                                          \end{bmatrix}
\end{align*}
and yielding the minimum of the probabilities induced by the variables we get
\[
	\underline{\pmin^I}(\Next (a \vee b)) =
	\begin{bmatrix} [0.4;0.7] \\ [0.3;0.7] \\ 1 \\ 1 \end{bmatrix} =
	\begin{bmatrix} 0.4 \\ 0.3 \\ 1 \\ 1 \end{bmatrix}
\]
thus $Sat(\prob_{>0.4}[\Next(a \vee b)]) = \{s \in S \mid \pmin^I(s, \Next(a \vee b)) > 0.4 \} = \{ s_2, s_3 \}$

\subsubsection{}

In order to check whether the IMDP satisfies the property
$\prob_{\le 0.5}[\Fin^{\le 3} a ]$ we will compute
\[
	Sat(\prob_{\le 0.5}[\Fin^{\le 3} a]) = \{ s \in S \mid \pmax^I(s, \Fin^{\le 3} a) \le 0.5 \}
\]
hence we will compute $\pmax^I(s, \Fin^{\le 3} a)$ for each state $s$ by
setting up a recursive equation, similarly to the MDP case.
The intervals will be handled as in the previous question but this time the
maximum probabilities will be extracted only at $k = 3$ if not trivial to do so
at every step as for MDPs.

First let's compute $S^{yes} = Sat(a) = \{ s_3 \}$, $S^{no} = \emptyset$ and
$S^? = \{ s_0, s_1, s_2 \}$ and then setup the recursive equation

\begin{align*}
	 & \pmax^I(s, \Fin^{\le k} a) = \\
	 & =
	\begin{cases}
		1                                                                                                                                                   & \text{if } s \in S^{yes}    \\
		0                                                                                                                                                   & \text{if } s \in S^{no}     \\
		0                                                                                                                                                   & \text{if } s \in S^?, k = 0 \\
		\max \left\{ \sum\limits_{s' \in S} \mu(s')\cdot\pmax^I(s', \Fin^{\le k-1} a) \mid (\alpha, \mathcal{M}) \in Steps(s), \mu \in \mathcal{M} \right\} & \text{otherwise}
	\end{cases}
\end{align*}

Hence getting:
\[
	\underline{\pmax^I}(\Fin^{\le 0} a) = \begin{bmatrix} 0 & 0 & 0 & 1 \end{bmatrix}
\]
\begin{align*}
	\underline{\pmax^I}(\Fin^{\le 1} a) & = Steps^\star \cdot \underline{\pmax^I}(\Fin^{\le 0} a)              \\
	                                    & = \begin{bmatrix}
		                                        0            & 0           & \mu_0^\alpha  & 1-\mu_0^\alpha \\
		                                        0            & \mu_0^\beta & 1-\mu_0^\beta & 0              \\
		                                        \hline
		                                        \mu_1^\alpha & 0           & 0             & 1-\mu_1^\alpha \\
		                                        \hline
		                                        0            & 0           & 1             & 0              \\
		                                        \hline
		                                        0            & 0           & 0             & 1              \\
	                                        \end{bmatrix} \cdot \begin{bmatrix} 0 \\ 0 \\ 0 \\ 1 \end{bmatrix} \\
	                                    & = \begin{bmatrix}
		                                        1-\mu_0^\alpha   \\
		                                        0                \\
		                                        \hline
		                                        1 - \mu_1^\alpha \\
		                                        \hline
		                                        0                \\
		                                        \hline
		                                        1
	                                        \end{bmatrix}                                                   \\
	                                    & = \begin{bmatrix}
		                                        1-\mu_0^\alpha   \\
		                                        1 - \mu_1^\alpha \\
		                                        0                \\
		                                        1
	                                        \end{bmatrix}
\end{align*}
\begin{align*}
	\underline{\pmax^I}(\Fin^{\le 2} a) & = Steps^\star \cdot \underline{\pmax^I}(\Fin^{\le 1} a)       \\
	                                    & = \begin{bmatrix}
		                                        0            & 0           & \mu_0^\alpha  & 1-\mu_0^\alpha \\
		                                        0            & \mu_0^\beta & 1-\mu_0^\beta & 0              \\
		                                        \hline
		                                        \mu_1^\alpha & 0           & 0             & 1-\mu_1^\alpha \\
		                                        \hline
		                                        0            & 0           & 1             & 0              \\
		                                        \hline
		                                        0            & 0           & 0             & 1              \\
	                                        \end{bmatrix} \cdot \begin{bmatrix}
		                                                            1-\mu_0^\alpha   \\
		                                                            1 - \mu_1^\alpha \\
		                                                            0                \\
		                                                            1
	                                                            \end{bmatrix} \\
	                                    & = \begin{bmatrix}
		                                        1 - \mu_0^\alpha                              \\
		                                        \mu_0^\beta(1 - \mu_1^\alpha)                 \\
		                                        \hline
		                                        \mu_1^\alpha(1-\mu_0^\alpha) + 1-\mu_1^\alpha \\
		                                        \hline
		                                        0                                             \\
		                                        \hline
		                                        1
	                                        \end{bmatrix}
	= \begin{bmatrix}
		  1 - \mu_0^\alpha              \\
		  \mu_0^\beta(1 - \mu_1^\alpha) \\
		  \hline
		  1 - \mu_0^\alpha \mu_1^\alpha \\
		  \hline
		  0                             \\
		  \hline
		  1
	  \end{bmatrix}
\end{align*}

\begin{align*}
	\underline{\pmax^I}(\Fin^{\le 3} a) & = Steps^\star \cdot \underline{\pmax^I}(\Fin^{\le 2} a)       \\
	                                    & = \begin{bmatrix}
		                                        0            & 0           & \mu_0^\alpha  & 1-\mu_0^\alpha \\
		                                        0            & \mu_0^\beta & 1-\mu_0^\beta & 0              \\
		                                        \hline
		                                        \mu_1^\alpha & 0           & 0             & 1-\mu_1^\alpha \\
		                                        \hline
		                                        0            & 0           & 1             & 0              \\
		                                        \hline
		                                        0            & 0           & 0             & 1              \\
	                                        \end{bmatrix} \cdot \begin{bmatrix}
		                                                            1 - \mu_0^\alpha              \\
		                                                            \mu_0^\beta(1 - \mu_1^\alpha) \\
		                                                            \hline
		                                                            1 - \mu_0^\alpha \mu_1^\alpha \\
		                                                            \hline
		                                                            0                             \\
		                                                            \hline
		                                                            1
	                                                            \end{bmatrix} \\
	                                    & = \begin{bmatrix}
		                                        1 - \mu_0^\alpha                                         \\
		                                        1 - \mu_0^\alpha                                         \\
		                                        \mu_0^\beta(1 - \mu_0^\alpha \mu_1^\alpha)               \\
		                                        \mu_0^\beta(1 - \mu_0^\alpha \mu_1^\alpha)               \\
		                                        \hline
		                                        \mu_1^\alpha(1-\mu_0^\alpha) + 1-\mu_1^\alpha            \\
		                                        \mu_1^\alpha\mu_0^\beta(1-\mu_1^\alpha) + 1-\mu_1^\alpha \\
		                                        \hline
		                                        0                                                        \\
		                                        \hline
		                                        1                                                        \\
	                                        \end{bmatrix}    \\
	                                    & = \begin{bmatrix}
		                                        1 - \mu_0^\alpha                                         \\
		                                        \mu_0^\beta(1 - \mu_0^\alpha \mu_1^\alpha)               \\
		                                        \hline
		                                        1- \mu_0^\alpha\mu_1^\alpha                              \\
		                                        \mu_1^\alpha\mu_0^\beta(1-\mu_1^\alpha) + 1-\mu_1^\alpha \\
		                                        \hline
		                                        0                                                        \\
		                                        \hline
		                                        1                                                        \\
	                                        \end{bmatrix}
\end{align*}
Considering the maximum probabilities achievable we get:
\begin{align*}
	\underline{\pmax^I}(\Fin^{\le 3} a)
	 & = \begin{bmatrix}
		     1 - \mu_0^\alpha                                         \\
		     \mu_0^\beta(1 - \mu_0^\alpha \mu_1^\alpha)               \\
		     \hline
		     1- \mu_0^\alpha\mu_1^\alpha                              \\
		     \mu_1^\alpha\mu_0^\beta(1-\mu_1^\alpha) + 1-\mu_1^\alpha \\
		     \hline
		     0                                                        \\
		     \hline
		     1                                                        \\
	     \end{bmatrix}
	 & = \begin{bmatrix}
		     0.3   \\
		     0.474 \\
		     \hline
		     0.79  \\
		     0.79  \\
		     \hline
		     0     \\
		     \hline
		     1     \\
	     \end{bmatrix}
	= \begin{bmatrix}
		  0.474 \\
		  0.79  \\
		  0     \\
		  1     \\
	  \end{bmatrix}
\end{align*}
Thus $Sat(\prob_{\le 0.5}[\Fin^{\le 3} a]) = \{ s \in S \mid \pmax^I(s, \Fin^{\le 3} a) \le 0.5 \} = \{ s_0, s_2 \}$.

\subsubsection{}

For computing $\pmax^I(s, \phi_1 \Until^{\le k} \phi_2])$ we would need to
setup a recursive equation as follows:
\begin{align*}
	 & \pmax^I(s, \phi_1 \Until^{\le k} \phi_2]) =                                                                                                  \\
	 & = \begin{cases}
		     1        & \text{if } s \in S^{yes}    \\
		     0        & \text{if } s \in S^{no}     \\
		     0        & \text{if } s \in S^?, k = 0 \\
		     \max \left\{
		     \sum\limits_{s' \in S} \mu(s')\cdot\pmax^I(s', \phi_1 \Until^{\le k-1} \phi_2) \mid (\alpha, \mathcal{M}) \in Steps(s), \mu \in \mathcal{M}
		     \right\} & \text{otherwise}
	     \end{cases}
\end{align*}
where $S^{yes} = Sat(\phi_2)$,
$S^{no} = S \setminus (Sat(\phi_1) \cup Sat(\phi_2))$ and
$S^? = S \setminus (Sat(\phi_1) \cup Sat(\phi_2))$.
It can be noted the similarity to the recursive equation for the MDP case which is the following
\begin{align*}
	 & \pmax(s, \phi_1 \Until^{\le k} \phi_2]) =                                                                     \\
	 & = \begin{cases}
		     1        & \text{if } s \in S^{yes}    \\
		     0        & \text{if } s \in S^{no}     \\
		     0        & \text{if } s \in S^?, k = 0 \\
		     \max \left\{
		     \sum\limits_{s' \in S} \mu(s')\cdot\pmax(s', \phi_1 \Until^{\le k-1} \phi_2) \mid (\alpha, \mu) \in Steps(s)
		     \right\} & \text{otherwise}
	     \end{cases}
\end{align*}
however in the IMDP case we need to consider all the possible intervals for the
transition probability function hence they arise in the recursive case.

In order to incorporate the intervals for the computations we would need to
define for eact action $n-1$ variables representing the actual transition
probability distribution of the action, where $n$ is the number of transitions
with an associated interval of probability distribution for the action.
The computations should be performed parametrically and then evaluated at the
last step in order to not prune out possible solutions.
