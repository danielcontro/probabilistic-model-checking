\subsection{}

\subsubsection{}

The DTMC D induced by the strategy of player 1 of always playing $rock$ is the
following:

\begin{center}
	\begin{tikzpicture}[->, >=latex, shorten >=1pt, line width=0.5 pt, node distance=3 cm, initial text=]
		\tikzstyle{every state}=[circle, black, draw]
		\tikzstyle{every loop}=[looseness=10]
		\node [state, initial] (s0)                                    {$s_0$};
		\node [state] (s1) [above right of=s0]  {$s_1$};
		\node [state] (s2) [below right of=s0]  {$s_2$};
		\node [state] (s3) [right of=s1]                      {$s_3$};
		\node [state] (s4) [right of=s2]                      {$s_4$};

		\node at ([shift={(0:1.2)}]s0) {$\{tie\}$};
		\node at ([shift={(300:1)}]s1) {$\{lead_1\}$};
		\node at ([shift={(60:1)}]s2) {$\{lead_2\}$};
		\node at ([shift={(270:0.9)}]s3) {$\{win_1\}$};
		\node at ([shift={(90:0.9)}]s4) {$\{win_2\}$};

		\path
		(s0)     edge [loop above] node [above]         {$\pi_r$} (s0)

		(s0.55)  edge              node [sloped, above] {$\pi_s$} (s1.215)
		(s1.235) edge              node [sloped, below] {$\pi_p$} (s0.35)
		(s1)     edge              node [sloped, above] {$\pi_s$} (s3)
		(s1)     edge [loop above] node [above]         {$\pi_r$} (s1)

		(s0.305) edge              node [sloped, below] {$\pi_p$} (s2.145)
		(s2.125) edge              node [sloped, above] {$\pi_s$} (s0.325)
		(s2)     edge              node [sloped, above] {$\pi_p$} (s4)
		(s2)     edge [loop below] node [below]         {$\pi_r$} (s2)

		(s3) edge [loop right] (s3)
		(s4) edge [loop right] (s4);
	\end{tikzpicture}
\end{center}

The (total) DFA A representing the property
\emph{"Both players lead at least once during the game"} is instead the
following:

\begin{center}
	\begin{tikzpicture}[->, >=latex, shorten >=1pt, line width=0.5 pt, node distance=3 cm, initial text=]
		\tikzstyle{every state}=[circle, black, draw]
		\tikzstyle{every loop}=[looseness=10]

		\node [state, initial]    (q0)                      {$q_0$};
		\node [state]             (q1) [above right of=q0]  {$q_1$};
		\node [state]             (q2) [below right of=q0]  {$q_2$};
		\node [state, accepting]  (q3) [above right of=q2]  {$q_3$};

		\path
		(q0) edge [loop above] node [above] {$\neg lead$} (q0)
		(q1) edge [loop above] node [above] {$\neg lead_2$} (q1)
		(q2) edge [loop below] node [below] {$\neg lead_1$} (q2)
		(q3) edge [loop right] node [right] {$true$} (q3)

		(q0) edge              node [sloped, above] {$lead_1$} (q1)
		(q0) edge              node [sloped, above] {$lead_2$} (q2)
		(q1) edge              node [sloped, above] {$lead_2$} (q3)
		(q2) edge              node [sloped, above] {$lead_1$} (q3)
		;
	\end{tikzpicture}
\end{center}

where:
\begin{align*}
	 & \neg lead = \{ tie, win_1, win_2 \}            \\
	 & \neg lead_1 = \{ tie, lead_2, win_1, win_2 \}  \\
	 & \neg lead_2 = \{ tie, lead_1, win_1, win_2 \}  \\
	 & true = \{ lead_1, lead_2, tie, win_1, win_2 \}
\end{align*}

\subsubsection{}

The product DTMC $D \otimes A$ is the following:

\begin{center}
	\begin{tikzpicture}[->, >=latex, shorten >=1pt, line width=0.5 pt, node distance=3 cm, initial text=]
		\tikzstyle{every state}=[circle, black, draw]
		\tikzstyle{every loop}=[looseness=10]

		\node [state, initial]    (s0q0)                        {$s_0q_0$};

		\node [state]             (s1q1) [above right of=s0q0]  {$s_1q_1$};
		\node [state]             (s3q1) [above of=s1q1]        {$s_3q_1$};
		\node [state]             (s0q1) [right of=s1q1]        {$s_0q_1$};
		\node [state]             (s2q3) [right of=s0q1]        {$s_2q_3$};

		\node [state]             (s2q2) [below right of=s0q0]  {$s_2q_2$};
		\node [state]             (s4q2) [below of=s2q2]        {$s_4q_2$};
		\node [state]             (s0q2) [right of=s2q2]        {$s_0q_2$};
		\node [state]             (s1q3) [right of=s0q2]        {$s_1q_3$};

		\node at ([shift={(90:1)}]s2q3) {$\{ acc \}$};
		\node at ([shift={(270:1)}]s1q3) {$\{ acc \}$};

		\path
		(s0q0) edge [loop above] node [above] {$\pi_r$} (s0q0)
		(s0q1) edge [loop above] node [above] {$\pi_r$} (s0q1)
		(s1q1) edge [loop below] node [below] {$\pi_r$} (s1q1)
		(s0q2) edge [loop below] node [below] {$\pi_r$} (s0q2)
		(s2q2) edge [loop above] node [above] {$\pi_r$} (s2q2)

		(s3q1) edge [loop right] node [right] {$1$} (s3q1)
		(s4q2) edge [loop right] node [right] {$1$} (s4q2)

		(s0q0) edge              node [sloped, above] {$\pi_s$} (s1q1)
		(s0q0) edge              node [sloped, above] {$\pi_p$} (s2q2)
		(s1q1) edge              node [sloped, above] {$\pi_s$} (s3q1)
		(s0q1) edge              node [sloped, above] {$\pi_p$} (s2q3)
		(s0q2) edge              node [sloped, above] {$\pi_s$} (s1q3)
		(s2q2) edge              node [sloped, above] {$\pi_p$} (s4q2)
		(s1q1.10) edge           node [sloped, above] {$\pi_p$} (s0q1.170)
		(s0q1.190) edge          node [sloped, below] {$\pi_s$} (s1q1.350)
		(s2q2.10) edge           node [sloped, above] {$\pi_s$} (s0q2.170)
		(s0q2.190) edge          node [sloped, below] {$\pi_p$} (s2q2.350)
		(s2q3) edge ++ (1.2,0)
		(s1q3) edge ++ (1.2,0)
		;
	\end{tikzpicture}
\end{center}

with every other state beyond $s_1q_3$ and $s_2q_3$ being accepting.

Now in order to compute the probability of the property being satisfied by the
DTMC, we need to compute
$Prob^D(s_0, P) = Prob^{D \otimes A}(s_0q_0, \Fin acc)$.

First we precompute the sets $S^{nO}$ and $S^{yes}$ for optimization purposes
using the Prob0 and Prob1 algorithms, obtaining:
\begin{align*}
	 & S^{no} = \{ s_3q_1, s_4q2 \}  \\
	 & S^{yes} = \{ s_2q_3, s_1q3 \}
\end{align*}
then we setup the system of linear equations:
\begin{align*}
	 & x_{s_3q_1} = x_{s_4q2} = 0                                          \\
	 & x_{s_2q_3} = x_{s_1q3} = 1                                          \\
	 & x_{s_0q_0} = \pi_r x_{s_0q_0} + \pi_s x_{s_1q_1} + \pi_p x_{s_2q_2} \\
	 & x_{s_1q_1} = \pi_r x_{s_1q_1} + \pi_s x_{s_3q_1} + \pi_p x_{s_0q_1} \\
	 & x_{s_0q_1} = \pi_r x_{s_0q_1} + \pi_s x_{s_1q_1} + \pi_p x_{s_2q_3} \\
	 & x_{s_2q_2} = \pi_r x_{s_2q_2} + \pi_s x_{s_0q_2} + \pi_p x_{s_4q_2} \\
	 & x_{s_0q_2} = \pi_r x_{s_0q_2} + \pi_s x_{s_1q_3} + \pi_p x_{s_2q_2}
\end{align*}
and solving we obtain:
\begin{align*}
	 & x_{s_3q_1} = x_{s_4q2} = 0 \\
	 & x_{s_2q_3} = x_{s_1q3} = 1 \\
	 & x_{s_0q_0} = 0.1613        \\
	 & x_{s_1q_1} = 0.8064        \\
	 & x_{s_0q_1} = 0.9677        \\
	 & x_{s_2q_2} = 0.0323        \\
	 & x_{s_0q_2} = 0.1935
\end{align*}
