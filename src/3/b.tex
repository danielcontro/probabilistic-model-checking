\subsection{}

\subsubsection{}

The statement:
\begin{center}
	\emph{"What is the maximum probability of player 1 winning within 3 rounds?"}
\end{center}
can be expressed with the following PCTL formula:
\[
	\prob_{\max=?}[\Fin^{\le 3} win_1]
\]

\subsubsection{}

Evaluating the formula $\prob_{\max=?}[\Fin^{\le 3} win_1]$ boils down to
compute the quantity $\pmax(s_0, \Fin^{\le 3} win_1)$ defined recursively as
follows:
\begin{align*}
	 & \pmax(s, \Fin^{\le k} \varphi) = \\
	 & =
	\begin{cases}
		1
		 & \text{if } s \in S^{yes}    \\
		0
		 & \text{if } s \in S^?, k = 0 \\
		\max \left\{ \sum\limits_{s' \in S} \mu(s') \cdot \pmax(s', \Fin^{\le k-1} \varphi) \mid (\alpha, \mu) \in Steps(s) \right\}
		 & \text{if }s \in S^?, k > 0
	\end{cases}
\end{align*}
where $S^{yes} = Sat(\varphi) = Sat(win_1) = \{ s_3 \}$ and
$S^? = S \setminus S^{yes} = \{ s_0, s_1, s_2, s_4 \}$.
In particular we will compute the vector
$\underline{\pmax}(\Fin^{\le 3} win_1)$ where
\[
	\underline{\pmax}(\Fin^{\le 0} win_1) = \begin{bmatrix} 0 & 0 & 0 & 1 & 0 \end{bmatrix}
\]
and
\begin{align*}
	\underline{\pmax}(\Fin^{\le k} win_1) & = Steps \cdot \underline{\pmax}(\Fin^{\le k-1} win_1)^T \\
	                                      & = \begin{bmatrix}
		                                          \pi_r & \pi_s & \pi_p & 0     & 0     \\
		                                          \pi_p & \pi_r & \pi_s & 0     & 0     \\
		                                          \pi_s & \pi_p & \pi_r & 0     & 0     \\
		                                          \hline
		                                          \pi_p & \pi_r & 0     & \pi_s & 0     \\
		                                          \pi_s & \pi_p & 0     & \pi_r & 0     \\
		                                          \pi_r & \pi_s & 0     & \pi_p & 0     \\
		                                          \hline
		                                          \pi_s & 0     & \pi_r & 0     & \pi_p \\
		                                          \pi_r & 0     & \pi_p & 0     & \pi_s \\
		                                          \pi_p & 0     & \pi_s & 0     & \pi_r \\
		                                          \hline
		                                          0     & 0     & 0     & 1     & 0     \\
		                                          \hline
		                                          0     & 0     & 0     & 0     & 1
	                                          \end{bmatrix} \cdot
	\begin{bmatrix}
		\pmax(s_0, \Fin^{\le k-1} win_1) \\
		\pmax(s_1, \Fin^{\le k-1} win_1) \\
		\pmax(s_2, \Fin^{\le k-1} win_1) \\
		\pmax(s_3, \Fin^{\le k-1} win_1) \\
		\pmax(s_4, \Fin^{\le k-1} win_1)
	\end{bmatrix}
\end{align*}
extracting at each iteration step the maximum value yielded by each state.
Hence we have:
\begin{align*}
	\underline{\pmax}(\Fin^{\le 1} win_1) & = Steps \cdot \underline{\pmax}(\Fin^{\le 0} win_1)^T                              \\
	                                      & = \begin{bmatrix}
		                                          \max \{ 0, 0, 0 \} & \max \{ \pi_s, \pi_r, \pi_p \} & \max \{ 0, 0, 0 \} & 1 & 0
	                                          \end{bmatrix} \\
	                                      & = \begin{bmatrix} 0 & \pi_r & 0 & 1 & 0 \end{bmatrix}
\end{align*}
\begin{align*}
	\underline{\pmax}(\Fin^{\le 2} win_1) & = Steps \cdot \underline{\pmax}(\Fin^{\le 1} win_1)^T                                                                         \\
	                                      & = \begin{bmatrix}
		                                          \max \{ \pi_s\pi_r, \pi_r^2, \pi_p\pi_r \} & \max \{ \pi_r^2 + \pi_s, \pi_p\pi_r + \pi_r, \pi_s\pi_r + \pi_p \} & 0 & 1 & 0
	                                          \end{bmatrix} \\
	                                      & = \begin{bmatrix} \pi_r^2 & \pi_p\pi_r + \pi_r & 0 & 1 & 0 \end{bmatrix}
\end{align*}
\begin{align*}
	\underline{\pmax}(\Fin^{\le 3} win_1) & = Steps \cdot \underline{\pmax}(\Fin^{\le 2} win_1)^T \\
	                                      & = \begin{bmatrix}
		                                          \pi_r^2(2\pi_p + 1)                   &
		                                          \pi_s\pi_r^2 + \pi_r(\pi_p^2+\pi_p+1) &
		                                          \pi_r^3                               &
		                                          1                                     &
		                                          0
	                                          \end{bmatrix}             \\
	                                      & = \begin{bmatrix}
		                                          0.6               &
		                                          0.890\overline{6} &
		                                          0.216             &
		                                          1                 &
		                                          0
	                                          \end{bmatrix}
\end{align*}
Thus $\prob_{\max=?}[\Fin^{\le 3} win_1] = 0.6$.

Since the strategy requested is relative to a step-bounded reachability
property in general it will need to be a finite-memory one.
In order to compute it we will need to perform backward computations and
selecting the argmax of $\underline{\pmax}$ (action that yields the maximum
probability) at each step starting from the goal state, namely:
\begin{itemize}
	\item $k=1$:
	      \begin{align*}
		       & \sigma_{\max}^1(s_0) = \_    \\
		       & \sigma_{\max}^1(s_1) = paper \\
		       & \sigma_{\max}^1(s_2) = \_
	      \end{align*}
	\item $k=2$:
	      \begin{align*}
		       & \sigma_{\max}^2(s_0) = paper \\
		       & \sigma_{\max}^2(s_1) = paper \\
		       & \sigma_{\max}^2(s_2) = \_
	      \end{align*}
	\item $k=3$:
	      \begin{align*}
		       & \sigma_{\max}^3(s_0) = paper \\
		       & \sigma_{\max}^3(s_1) = paper \\
		       & \sigma_{\max}^3(s_2) = paper
	      \end{align*}
\end{itemize}

Thus obtaining $\sigma_{\max}(s) = paper$ for the property analyzed.


\subsubsection{}

In a setting where player 1 know player 2's strategy and plays optimally
against it, in order to minimise the maximum probability of player 1 winning
player 2 will need to minimise the maximum of mass of probability of player 1
winning that leaks out at every step.
Hence player 2 will need to adopt a strategy where every
action is be equiprobable, thus the optimal strategy for player 2 will be
$(\frac{1}{3}, \frac{1}{3}, \frac{1}{3})$.
